% \iffalse
%<*driver>
\ProvidesFile{ut-thesis.dtx}
%</driver>
%<class>\NeedsTeXFormat{LaTeX2e}[1999/12/01]
%<class>\ProvidesClass{ut-thesis}
%<*class>
[2020/09/01 v3.0.a University of Toronto thesis class]
%</class>
%
%<*driver>
\documentclass[10pt]{ltxdoc}
\usepackage[osf]{mathpazo}
\usepackage[margin=4cm]{geometry}
\usepackage{xcolor}
\definecolor{code} {HTML}{990033}
\definecolor{link} {HTML}{000066}
\let\ottfamily\ttfamily
\renewcommand{\ttfamily}{\color{code}\ottfamily}
\renewcommand{\MacroFont}{\ttfamily\color{code}}
\usepackage[colorlinks,linkcolor=link]{hyperref}
\setlength{\skip\footins}{4ex}
\setlength{\parindent}{0pt}
\setlength{\parskip}{6pt}
\begin{document}
  \DocInput{ut-thesis.dtx}
\end{document}
%</driver>
% \fi
%
% \GetFileInfo{ut-thesis.dtx}
%
% \title{The \texttt{ut-thesis} class
%   \thanks{\fileversion~[\filedate] CTAN repository:
%   \href{https://ctan.org/pkg/ut-thesis}
%        {\texttt{https://ctan.org/pkg/ut-thesis}}}}
% \author{
%   Francois Pitt,
%   Jesse Knight
%     \thanks{maintainer, contact:
%     \href{mailto:jesse.knight@mail.utoronto.ca}
%          {\texttt{jesse.knight@mail.utoronto.ca}}}
% }
%
% \maketitle
%
% \begin{abstract}\noindent
%   The |ut-thesis| document class and template implements
%   the requirements of the University of Toronto School of Graduate Studies (SGS),
%   as of Winter 2020.
% \end{abstract}
%
% \tableofcontents
% \clearpage
%
% \section{Usage}
% |\documentclass{ut-thesis}|\\
% |\documentclass[...options...]{ut-thesis}|
%
% The default settings produce a final copy, ready for submission to
% the School of Graduate Studies (SGS) at the University of Toronto:
% single-sided, "normal" margins (see below), one-and-a-half spaced
% with single-spaced notes.
%
% \subsection{Options}
%
%  - Any standard option for the LaTeX2e |book| class, including
%    |10pt|, |11pt|, |12pt|, |oneside|, |twoside|, etc.
% 
%  - |narrowmargins|, |normalmargins|, |widemargins|, or
%    |extrawidemargins|:  Set the size of the margins, as follows:
%     . |narrow|: 1 1/4" on the left, 3/4" on all other sides,
%       headers \& footers 3/8" from body
%       (these are the minimum values required by SGS);
%     . |normal|: 1 1/4" on the left, 1" on all other sides,
%       headers \& footers 1/2" from body;
%     . |wide|: 1 1/4" on all sides,
%       headers \& footers 5/8" from body;
%     . |extrawide|: 1 1/2" on all sides,
%       headers \& footers 3/4" from body.
%    If you have more than just a few marginal notes, it is recommended
%    that you use at least |wide| margins.  For other settings, use the
%    |\geometry| command (see the template for details).
% 
%  - |singlespaced|, |oneandahalfspaced|, or |doublespaced|:  Set the
%    entire document's default line spacing (except for notes, which
%    are single-spaced by default).  For other settings, use the
%    |\setstretch| command (see the template for details).
% 
%  - |singlespacednotes| or |standardspacednotes|:  Set line spacing
%    for footnotes and marginal notes: either single-spaced or the same
%    as the rest of the document.
% 
%  - |cleardoublepagestyleempty|, |cleardoublepagestyleplain|, or
%    |cleardoublepagestylestandard|:  Set the page style for all
%    "cleared" pages (empty pages inserted in two-sided documents to
%    put the next page on the right-hand side) to either |empty|,
%    |plain|, or whatever style is in effect when the page is cleared
%    (the default).
% 
%  - |draft|:  Produce a draft copy (double-sided, double-spaced,
%    normal margins, with the word "DRAFT" printed at all four corners
%    of every page).
% 
% Note that these options can be used to override the default or draft
% document settings, so that it is possible, for example, to create a
% double-sided final copy, or a 1 1/2-spaced draft copy with wide
% margins, etc.  You may use standard LaTeX packages to tailor the
% layout and formatting in other ways.
% Also note that when producing double-sided documents while \emph{not} in
% draft mode, new chapters and preliminary sections will always start
% on a right-hand page under the default settings (inserting a blank
% page if needed).  This can be overridden by using the |openany| or
% |openright| options.  To achieve this effect for individual sections
% or chapters, use |\cleardoublepage| -- or one of the more specific
% |\clearemptydoublepage|, |\clearplaindoublepage|, |\clearthesisdoublepage|,
% or |\clearstandarddoublepage| (see below for details).
% 
% \subsection{Environments \& Commands}
% 
%  * |\degree{...}|:  (preamble only; REQUIRED)
%    Specify the name of the degree (e.g., "Doctor of Philosophy").
% 
%  * |\department{...}|:  (preamble only; REQUIRED)
%    Specify the name of the graduate department.
% 
%  * |\gradyear{...}|:  (preamble only; REQUIRED)
%    Specify the year of graduation (defaults to current year).
% 
%  * |\author{...}|:  (preamble only; REQUIRED)
%    Specify the name of the author.
% 
%  * |\title{...}|:  (preamble only; REQUIRED)
%    Specify the title of the thesis.
% 
%  - |\begin{preliminary}...\end{preliminary}|:
%    Delimit head matter (title page, abstract, table of contents,
%    lists of tables and figures, etc.): set the page style and
%    numbering for the preliminary sections and reset them for the main
%    document.
% 
%     - |\maketitle|:
%       Generate the title page from the information supplied in the
%       preamble.
% 
%     - |\begin{abstract}...\end{abstract}|:
%       Generate the abstract page, double-sided.  (According to SGS
%       guidelines, this must immediately follow the title page.)
% 
%     - |\begin{dedication}...\end{dedication}|:
%       Generate a dedication section, if needed (just a paragraph
%       formatted flush right).
% 
%     - |\begin{acknowledgements}...\end{acknowledgements}|:
%       Generate an acknowledgements section, if needed.
% 
%    Note that neither the |dedication| nor the |acknowledgements| are
%    put on a separate page by default (use |\newpage| to do this
%    explicitly).  Also note that the table of contents, list of
%    tables, and list of figures can be generated using the usual LaTeX
%    commands.
% 
%  - |\begin{longquote}...\end{longquote}|:
%    Single-spaced version of the |quote| environment.
% 
%  - |\begin{longquotation}...\end{longquotation}|:
%    Single-spaced version of the |quotation| environment.
% 
%  - |\clearemptydoublepage|, |\clearplaindoublepage|,
%    |\clearthesisdoublepage|:
%    Same as |\cleardoublepage| except that cleared pages have style
%    |empty|, |plain|, or |thesis| respectively.
% 
%  - |\clearstandarddoublepage|:
%    Same as the original |\cleardoublepage| (cleared pages use the style
%    currently in effect) -- used to override the effects of options
%    |cleardoublepagestyleempty| or |cleardoublepagestyleplain|.
% 
% The companion file |ut-thesis.tex| contains a skeleton illustrating
% the use of this class.
%
% \section{Implementation}
% \subsection{Initial Code}

%% Switch for testing draft mode (toggled by |draft| option).
%    \begin{macrocode}
\newif\if@draft
\@draftfalse
%    \end{macrocode}

%% Switch for testing current page style.
%    \begin{macrocode}
\newif\if@thesispagestyle
\@thesispagestyletrue
%    \end{macrocode}

%% Save original definitions of footnote and marginal note macros and
%% lengths, to be able to reset them below as needed (when changing
%% between single-spaced and standard-spaced notes).
%    \begin{macrocode}
\let\@thesis@footnotetext\@footnotetext
\let\@thesis@mpfootnotetext\@mpfootnotetext
\let\@thesis@marginparreset\@marginparreset
\newlength{\@thesisfootnotesep}
\newlength{\@thesismarginparpush}
\AtBeginDocument
  {\setlength\@thesisfootnotesep{\footnotesep}
   \setlength\@thesismarginparpush{\marginparpush}}
%    \end{macrocode}

%% Save original definition of |\cleardoublepage|.
%    \begin{macrocode}
\let\clearstandarddoublepage\cleardoublepage
%    \end{macrocode}

%% The |\singlespacing| macro from |setspace| includes some vertical space
%% (to make it easier to change line spacing within the document).
%% Unfortunately, this has undesirable side-effects within macros, so we
%% define our own replacement here for use within the class.
%    \begin{macrocode}
\newcommand*{\singlespacingnoskip}{\setstretch{\setspace@singlespace}}
%    \end{macrocode}

% \subsection{Option Declaration}

%% |draft| option: change default document settings.
%    \begin{macrocode}
\DeclareOption{draft}{\@drafttrue
   \newcommand*{\tlDRAFT}%
     {\raisebox{ 3ex}[0pt][0pt]{\llap{\sffamily\scriptsize DRAFT\ \ }}}
   \newcommand*{\trDRAFT}%
     {\raisebox{ 3ex}[0pt][0pt]{\rlap{\sffamily\scriptsize \ \ DRAFT}}}
   \newcommand*{\blDRAFT}%
     {\raisebox{-3ex}[0pt][0pt]{\llap{\sffamily\scriptsize DRAFT\ \ }}}
   \newcommand*{\brDRAFT}%
     {\raisebox{-3ex}[0pt][0pt]{\rlap{\sffamily\scriptsize \ \ DRAFT}}}
   \ExecuteOptions{doublespaced}
   \PassOptionsToClass{draft,twoside,openany}{book}}
%    \end{macrocode}

%% Margin options.
%    \begin{macrocode}
\DeclareOption{narrowmargins}{\AtEndOfClass % 1 1/4" left, 3/4" others
  {\geometry{margin=.75in,left=1.25in,headsep=.375in-\headheight,
             footskip=.375in,marginparwidth=.5in,marginparsep=.125in}}}
\DeclareOption{normalmargins}{\AtEndOfClass % 1 1/4" left, 1" others
  {\geometry{margin=1in,left=1.25in,headsep=.5in-\headheight,
             footskip=.5in,marginparwidth=.75in,marginparsep=.125in}}}
\DeclareOption{widemargins}{\AtEndOfClass % 1 1/4" all around
  {\geometry{margin=1.25in,headsep=.625in-\headheight,
             footskip=.625in,marginparwidth=.75in,marginparsep=.25in}}}
\DeclareOption{extrawidemargins}{\AtEndOfClass % 1 1/2" all around
  {\geometry{margin=1.5in,headsep=.75in-\headheight,
             footskip=.75in,marginparwidth=1in,marginparsep=.25in}}}
%    \end{macrocode}

%% Line Spacing options.
%    \begin{macrocode}
\DeclareOption{singlespaced}{\AtEndOfClass{\singlespacingnoskip}}
\DeclareOption{onehalfspaced}{\AtEndOfClass{\onehalfspacing}}
\DeclareOption{doublespaced}{\AtEndOfClass{\doublespacing}}
%    \end{macrocode}

%% Line spacing for notes.
%    \begin{macrocode}
\DeclareOption{singlespacednotes}{\AtBeginDocument
  {\setlength\footnotesep{\@thesisfootnotesep}
   \setlength\marginparpush{\@thesismarginparpush}
   \renewcommand{\@footnotetext}[1]%
     {\@thesis@footnotetext{#1\singlespacingnoskip}}
   \renewcommand{\@mpfootnotetext}[1]%
     {\@thesis@mpfootnotetext{#1\singlespacingnoskip}}
   \renewcommand*{\@marginparreset}%
     {\@thesis@marginparreset\singlespacingnoskip}}}
\DeclareOption{standardspacednotes}{\AtBeginDocument
  {\setlength\footnotesep{\baselineskip-\@thesisfootnotesep}
   \setlength\marginparpush{\baselineskip-\@thesismarginparpush}
   \let\@footnotetext\@thesis@footnotetext
   \let\@mpfootnotetext\@thesis@mpfootnotetext
   \let\@marginparreset\@thesis@marginparreset}}
%    \end{macrocode}

%% Page styles for cleared pages.
%    \begin{macrocode}
\DeclareOption{cleardoublepagestyleempty}
  {\AtEndOfClass{\let\cleardoublepage\clearemptydoublepage}}
\DeclareOption{cleardoublepagestyleplain}
  {\AtEndOfClass{\let\cleardoublepage\clearplaindoublepage}}
\DeclareOption{cleardoublepagestylestandard}
  {\AtEndOfClass{\let\cleardoublepage\clearstandarddoublepage}}
%    \end{macrocode}

%% All other options are passed to the base class directly.
%    \begin{macrocode}
\DeclareOption*{\PassOptionsToClass{\CurrentOption}{book}}
%    \end{macrocode}

% \subsection{Option Execution}

%% Default settings: standard options followed by ut-thesis options.
%    \begin{macrocode}
\ExecuteOptions{letterpaper,oneside,openright}
\ExecuteOptions{normalmargins,onehalfspaced,singlespacednotes}
%    \end{macrocode}

%% Process options.
%    \begin{macrocode}
\ProcessOptions
%    \end{macrocode}

% \subsection{Package Loading}

%% Load base class using current setting for basic options.
%    \begin{macrocode}
\LoadClass{book}
%    \end{macrocode}

%% To set/change page layout.
%    \begin{macrocode}
\RequirePackage{calc}
\RequirePackage{geometry}
%    \end{macrocode}

%% To set/change line spacing.
%    \begin{macrocode}
\RequirePackage{setspace}
%    \end{macrocode}
%
% \subsection{Author Information}
% Getting the user inputs.
%
%    \begin{macrocode}
\renewcommand*{\author}  [1]{\gdef\@author{#1}}
\renewcommand*{\title}   [1]{\gdef\@title{#1}}
\newcommand*{\degree}    [1]{\gdef\@degree{#1}}
\newcommand*{\department}[1]{\gdef\@department{#1}}
\newcommand*{\gradyear}  [1]{\gdef\@gradyear{#1}}
%    \end{macrocode}
%
% Setting default values that will hopefully be overwritten.
%
%    \begin{macrocode}
\author    {(author)}
\title     {(title)}
\degree    {(degree)}
\department{(department)}
\gradyear  {(gradyear)}
%    \end{macrocode}

% \subsection{Front Matter}

% \subsubsection{Title Page}
% We don't enforce firm distances between lines,
% but use |\vfill| to stretch and fill the space evenly,
% except for a double-sized gap after the author name.
% There is one part of space above the title,
% while the copyright is pushed all the way to the bottom.
%
%    \begin{macrocode}
\renewcommand*{\maketitle}%
  {\thispagestyle{empty}
   \large
   \begin{center}
      \singlespacing
      \null
      \vfill
      \textsc{\@title}
      \vfill
      by
      \vfill
      {\@author}
      \vfill
      \vfill
      A thesis submitted in conformity with the requirements\\
      for the degree of {\@degree}\\[1ex]
      Graduate Department of {\@department}\\
      University of Toronto\\
      \vfill
      {\copyright} Copyright {\@gradyear} by {\@author}
   \end{center}
   \cleardoublepage}
%    \end{macrocode}

% \subsubsection{Abstract Page}
% The abstract is an environment, but it creates its own page
% (and possibly an extra empty page if using |twoside|).
% The author and title info is centered and singlespaced.
% The word ``Abstract'' uses the |\section*| style, without any numbering.
% The abstract content is doublespaced.
%
%    \begin{macrocode}
\newenvironment*{abstract}%
  {\thispagestyle{plain}
   \begin{center}
     \singlespacing
      {\@title}\\[2ex]
      {\@author}\\
      {\@degree}\\[1ex]
      Graduate Department of {\@department}\\
      University of Toronto\\
      {\@gradyear}\\
      \section*{Abstract}
   \end{center}
   \begingroup
   \doublespacing}%
  {\endgroup\cleardoublepage}
%    \end{macrocode}

%% |\begin{dedication}...\end{dedication}| formats a dedication section
%% (*not* on a separate page -- just a paragraph formatted flush right).
%    \begin{macrocode}
\newenvironment*{dedication}%
  {\begin{flushright}}%
  {\end{flushright}}
%    \end{macrocode}

%% |\begin{acknowledgements}...\end{acknowledgements}| formats an
%% acknowledgements section (*not* on a separate page).
%    \begin{macrocode}
\newenvironment*{acknowledgements}%
  {\begin{center}
      \section*{Acknowledgements}
   \end{center}
   \begingroup\noindent}%
  {\par\endgroup}
%    \end{macrocode}

%% Redefine |\thebibliography| environment so that it generates headers
%% in the same style as the rest of the document.
%    \begin{macrocode}
\let\@thesisthebibliography\thebibliography
\renewcommand*{\thebibliography}[1]{\@thesisthebibliography{#1}
   \if@thesispagestyle\@mkboth{\textsc{\bibname}}{\textsc{\bibname}}\fi}
%    \end{macrocode}

%% Variations of |\cleardoublepage| that explicitly set the pagestyle of
%% any inserted blank page.
%    \begin{macrocode}
\newcommand*{\clearemptydoublepage}%
  {{\pagestyle{empty}\clearstandarddoublepage}}
\newcommand*{\clearplaindoublepage}%
  {{\pagestyle{plain}\clearstandarddoublepage}}
\newcommand*{\clearthesisdoublepage}%
  {{\pagestyle{thesis}\clearstandarddoublepage}}
%    \end{macrocode}

%% Single-spaced quotes and quotations.
%    \begin{macrocode}
\newenvironment*{longquote}%
  {\begin{quote}\singlespacingnoskip}{\end{quote}}
\newenvironment*{longquotation}%
  {\begin{quotation}\singlespacingnoskip}{\end{quotation}}
%    \end{macrocode}

% \subsubsection{Page Styles}

%% Redefine all four standard page styles (empty, plain, headings,
%% myheadings), based on the definitions in |book|, so that they
%% conform to the SGS guidelines (and include draft information if
%% applicable).  Then, define a new pagestyle |thesis|.

%% TODO: Get rid of copy-pasted definitions for pagestyles?

%% Pagestyle |empty|.
%    \begin{macrocode}
\renewcommand*{\ps@empty}%
  {\@thesispagestylefalse
   \let\@mkboth\@gobbletwo
   \def\@oddfoot{\if@draft\blDRAFT\hfil
      {\slshape\small\today}\hfil\brDRAFT\fi}%
   \let\@evenfoot\@oddfoot
   \def\@oddhead{\if@draft\tlDRAFT\hfil
      {\slshape\small\today}\hfil\trDRAFT\fi}%
   \let\@evenhead\@oddhead}
%    \end{macrocode}

%% Pagestyle |plain|.
%    \begin{macrocode}
\renewcommand*{\ps@plain}%
  {\@thesispagestylefalse
   \let\@mkboth\@gobbletwo
   \def\@oddfoot{\if@draft\blDRAFT\fi\hfil
      \thepage\hfil\if@draft\brDRAFT\fi}%
   \let\@evenfoot\@oddfoot
   \def\@oddhead{\if@draft\tlDRAFT\hfil
      {\slshape\small\today}\hfil\trDRAFT\fi}%
   \let\@evenhead\@oddhead}
%    \end{macrocode}

%% Pagestyle |headings|.
%    \begin{macrocode}
\if@twoside % two-sided printing
\renewcommand*{\ps@headings}%
  {\@thesispagestylefalse
   \let\@mkboth\markboth
   \def\@oddfoot{\if@draft\blDRAFT\hfil
      {\slshape\small\today}\hfil\brDRAFT\fi}%
   \let\@evenfoot\@oddfoot
   \def\@oddhead{\if@draft\tlDRAFT\fi{\slshape\rightmark}\hfil
      \thepage\if@draft\trDRAFT\fi}%
   \def\@evenhead{\if@draft\tlDRAFT\fi\thepage\hfil
      {\slshape\leftmark}\if@draft\trDRAFT\fi}%
   \def\chaptermark##1{\markboth
      {\MakeUppercase{\ifnum \c@secnumdepth >\m@ne
         \@chapapp\ \thechapter. \ \fi ##1}}{}}%
   \def\sectionmark##1{\markright
      {\MakeUppercase{\ifnum \c@secnumdepth >\z@
         \thesection. \ \fi ##1}}}}
\else % one-sided printing
\renewcommand*{\ps@headings}%
  {\@thesispagestylefalse
   \let\@mkboth\markboth
   \def\@oddfoot{\if@draft\blDRAFT\hfil
      {\slshape\small\today}\hfil\brDRAFT\fi}%
   \def\@oddhead{\if@draft\tlDRAFT\fi{\slshape\rightmark}\hfil
      \thepage\if@draft\trDRAFT\fi}%
   \def\chaptermark##1{\markright
      {\MakeUppercase{\ifnum \c@secnumdepth >\m@ne
         \@chapapp\ \thechapter. \ \fi ##1}}}}
\fi%@twoside
%    \end{macrocode}

%% Pagestyle |myheadings|.
%    \begin{macrocode}
\renewcommand*{\ps@myheadings}%
  {\@thesispagestylefalse
   \let\@mkboth\@gobbletwo
   \def\@oddfoot{\if@draft\blDRAFT\hfil
      {\slshape\small\today}\hfil\brDRAFT\fi}%
   \let\@evenfoot\@oddfoot
   \def\@oddhead{\if@draft\tlDRAFT\fi{\slshape\rightmark}\hfil
      \thepage\if@draft\trDRAFT\fi}%
   \def\@evenhead{\if@draft\tlDRAFT\fi\thepage\hfil
      {\slshape\leftmark}\if@draft\trDRAFT\fi}%
   \let\chaptermark\@gobble\let\sectionmark\@gobble}
%    \end{macrocode}

%% Pagestyle |thesis| (based on |headings|).
%    \begin{macrocode}
\if@twoside % two-sided printing
\newcommand*{\ps@thesis}%
  {\@thesispagestyletrue
   \let\@mkboth\markboth
   \def\@oddfoot{\if@draft\blDRAFT\hfil
      {\slshape\small\today}\hfil\brDRAFT\fi}%
   \let\@evenfoot\@oddfoot
   \def\@oddhead{\if@draft\tlDRAFT\fi{\slshape\rightmark}\hfil
      \thepage\if@draft\trDRAFT\fi}%
   \def\@evenhead{\if@draft\tlDRAFT\fi\thepage\hfil
      {\slshape\leftmark}\if@draft\trDRAFT\fi}%
   \def\chaptermark##1{\markboth
      {\textsc{\ifnum \c@secnumdepth >\m@ne
         \@chapapp\ \thechapter. \ \fi ##1}}{}}%
   \def\sectionmark##1{\markright
      {\textsc{\ifnum \c@secnumdepth >\z@
         \thesection. \ \fi ##1}}}}
\else % one-sided printing
\newcommand*{\ps@thesis}%
  {\@thesispagestyletrue
   \let\@mkboth\markboth
   \def\@oddfoot{\if@draft\blDRAFT\hfil
      {\slshape\small\today}\hfil\brDRAFT\fi}%
   \def\@oddhead{\if@draft\tlDRAFT\fi{\slshape\rightmark}\hfil
      \thepage\if@draft\trDRAFT\fi}%
   \def\chaptermark##1{\markright
      {\textsc{\ifnum \c@secnumdepth >\m@ne
         \@chapapp\ \thechapter. \ \fi ##1}}}}
\fi%@twoside
%    \end{macrocode}

%% Default page style.
%    \begin{macrocode}
\pagestyle{thesis}
%    \end{macrocode}
\endinput